\documentclass[11pt]{article}

\usepackage[T1]{fontenc}
\usepackage{lmodern}
\usepackage{amsmath,amssymb,amsthm}
\usepackage{geometry}
\usepackage{hyperref}

\geometry{margin=1in}

\theoremstyle{definition}
\newtheorem{definition}{Definition}[section]

\theoremstyle{plain}
\newtheorem{theorem}{Theorem}[section]

\theoremstyle{remark}
\newtheorem{remark}{Remark}[section]

\title{On the Structural Limits of Catalog-Based Scientific Discovery}
\author{Inacio F. Vasquez\\Independent Researcher}
\date{February 2026}

\begin{document}
\maketitle

\begin{abstract}
We propose a thermodynamic limit to catalog-based scientific discovery, modeling
discovery as a constrained information refinement process. We introduce a discovery
capacity invariant measuring the efficiency of novel object extraction per bit of physical
information. We prove that under locality and finite information constraints, discovery
necessarily exhibits universal saturation behavior. An executable instantiation using
NEOWISE and AllWISE infrared survey data is presented as an empirical test of this
epistemic law.
\end{abstract}

\section{Conceptual Framing}

Scientific discovery increasingly occurs through large observational catalogs:
astronomical surveys, genomic databases, sensor networks, and automated experiments.
In such systems, discovery is no longer the identification of isolated phenomena, but
the progressive refinement of structured information spaces.

This motivates a foundational question:

\emph{Is there a universal physical limit to catalog-based discovery?}

We argue that discovery is an information refinement process subject to intrinsic
saturation laws imposed by the physical nature of information itself.

\section{Discovery as Refinement}

\begin{definition}[Discovery System]
A discovery system is a tuple $(\mathcal D,\mathcal C,\mathcal T)$ where:
\begin{itemize}
\item $\mathcal D$ is a finite data stream,
\item $\mathcal C$ is a finite configuration space of candidate objects,
\item $\mathcal T$ is a family of admissible physical tests.
\end{itemize}
Discovery proceeds by iteratively excluding configurations via tests, producing a
decreasing sequence $\mathcal C_k \subseteq \mathcal C$.
\end{definition}

\begin{definition}[Discovery Entropy]
Let $\mu_k$ be the distribution on $\mathcal C_k$. Define
\[
H_k = -\sum_{c\in\mathcal C_k}\mu_k(c)\log\mu_k(c).
\]
\end{definition}

\section{Discovery Capacity Invariant}

\begin{definition}[Discovery Capacity]
Define the discovery capacity at step $k$ by
\[
\Gamma_k = \frac{\Delta O_k}{\Delta I_k},
\]
where $\Delta O_k$ is the number of newly identified objects and $\Delta I_k$ is the
physical information gained.
\end{definition}

\begin{theorem}[Discovery Saturation]
If tests are locality-preserving and information extraction is physically capacity bounded,
then $\Gamma_k \to 0$ as $k\to\infty$.
\end{theorem}

\begin{proof}
Since $\mathcal C$ is finite,
\[
\sum_{k=1}^\infty \Delta O_k \le |\mathcal C|,
\]
hence $\Delta O_k \to 0$.

By thermodynamic irreversibility (Landauer’s principle), the physical information cost
per step satisfies $\Delta I_k \ge C_0 > 0$. Therefore,
\[
\Gamma_k = \frac{\Delta O_k}{\Delta I_k} \longrightarrow 0.
\]
\end{proof}

\section{Physical Interpretation}

Information is physical. Any discovery process must be implemented by systems with
finite energy, entropy, and storage capacity. Landauer’s principle and the Bekenstein
bound imply finite distinguishability, yielding saturation as a physical law.

\begin{remark}
Discovery saturation is a thermodynamic consequence of finite information capacity,
not a sociological limitation.
\end{remark}

\section{Empirical Test: NEOWISE / AllWISE}

We test the predicted saturation structure using NEOWISE and AllWISE survey data.

\subsection{Information Proxy}

Using signal threshold $\tau=7$, define
\[
I_k = 4\log_2 |N_k|,
\]
corresponding to four floating-point observables per detection.

\subsection{Moving Object Yield}

Let $O_k^{\mathrm{new}}$ denote newly identified tracklets at step $k$.
Empirically, $\Gamma_k$ decreases rapidly over time, consistent with the theorem.

\section{Conclusion}

Catalog-based discovery obeys universal physical saturation laws. As discovery proceeds,
novel object yield diminishes while information cost remains bounded below. Scientific
progress asymptotically shifts from discovery to refinement.

\begin{thebibliography}{9}

\bibitem{Shannon}
C.~E.~Shannon,
\emph{A Mathematical Theory of Communication},
Bell System Technical Journal 27 (1948).

\bibitem{Landauer}
R.~Landauer,
\emph{Irreversibility and Heat Generation in the Computing Process},
IBM J. Res. Dev. 5 (1961).

\bibitem{Bekenstein}
J.~D.~Bekenstein,
\emph{Information in the Holographic Universe},
Scientific American 289 (2003).

\bibitem{Lloyd}
S.~Lloyd,
\emph{Ultimate Physical Limits to Computation},
Nature 406 (2000).

\end{thebibliography}

\end{document}

