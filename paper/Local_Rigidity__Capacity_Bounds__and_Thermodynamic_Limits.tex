\documentclass[11pt]{article}

\usepackage[T1]{fontenc}
\usepackage{lmodern}
\usepackage{amsmath,amssymb,amsthm}
\usepackage{geometry}
\usepackage{hyperref}
\geometry{margin=1in}

\title{Local Rigidity, Capacity Bounds, and Thermodynamic Limits}
\author{Inacio F. Vasquez}
\date{January 2026}

\theoremstyle{definition}
\newtheorem{definition}{Definition}[section]

\theoremstyle{plain}
\newtheorem{theorem}[definition]{Theorem}
\newtheorem{lemma}[definition]{Lemma}

\begin{document}
\maketitle

\begin{abstract}
We study universal obstructions to inference and control through local interfaces
under finite informational capacity. Across finite model theory, algorithms, and
physics, local mechanisms are routinely used to extract global structure. Under
minimal admissibility assumptions—finite transcript capacity and thermodynamic
calibration—any such mechanism satisfies a sharp rigidity principle: whenever
information demand exceeds capacity, either irreducible error persists or a
detectable regime-exit signature must occur. This yields a universal ceiling on
extractable mutual information and formalizes the Viotropic Wall: beyond capacity,
additional internal structure becomes operationally null. All conditional statements
are explicitly labeled.
\end{abstract}

\section{Background and Motivation}

Locality organizes inference in multiple disciplines. In finite model theory,
bounded-variable logics and Ehrenfeucht--Fra\"{\i}ss\'{e} games limit global inference
from neighborhoods. In computation, refinement algorithms fail on instances
requiring long-range consistency. In physics, measurement is mediated by finite
interfaces constrained by energy, geometry, and entropy production.

\section{Model: Interfaces and Transcripts}

\begin{definition}[Interface]
Let $X$ be an unknown system state. An interface $I$ produces an interactive transcript
\[
Y_{1:T}=(Y_1,\dots,Y_T)
\]
according to update rules
\[
Y_t=\Phi_t(\mathrm{Obs}_t(X),Y_{<t}),
\]
with finite internal memory.
\end{definition}

\begin{definition}[Transcript Capacity]
The transcript capacity over horizon $T$ is
\[
TC(T):=\sup I(X;Y_{1:T}),
\]
where the supremum ranges over admissible strategies of $I$.
\end{definition}

\paragraph{Finite Capacity (FC).}
For any declared admissible regime, $TC(T)<\infty$ and scales at most linearly in $T$.

\section{Landauer Calibration}

\paragraph{Landauer Calibration.}
If the interface operates at temperature $\Theta$ and dissipates heat rate $\dot Q(t)$,
then
\[
TC(T)\le \frac{1}{k_B\Theta\ln 2}\int_0^T \dot Q(t)\,dt .
\]

\begin{lemma}[Landauer Mutual Information Ceiling]
For any interactive channel implemented by a finite-memory interface,
\[
I(X;Y_{1:T})\le \frac{1}{k_B\Theta\ln 2}\int_0^T \dot Q(t)\,dt .
\]
\end{lemma}

\begin{proof}
$I(X;Y_{1:T})\le H(Y_{1:T})\le H(M_T)$, where $M_T$ is the internal memory.
Each logically irreversible update erases at least one bit and dissipates
$\ge k_B\Theta\ln 2$ heat. Summing over updates yields the bound.
\end{proof}

\section{Demand--Capacity Rigidity}

\begin{definition}[Information Demand]
Let $\mathrm{Dem}_\eta(T)$ denote the mutual information required to achieve task
accuracy $\eta$ by time $T$.
\end{definition}

\begin{theorem}[Rigidity Dichotomy]
Assume \textbf{FC}. If $\mathrm{Dem}_\eta(T)>TC(T)$, then no admissible strategy
achieves accuracy $\eta$ by time $T$. Any apparent success implies a regime exit
witnessed by at least one of:
(i) transcript escape (nonlocal information),
(ii) defect persistence (certified information gap),
(iii) instability (variance blow-up).
\end{theorem}

\section{Viotropic Wall}

\begin{definition}[Operational Mutual Information]
For any observable $O=f(Y_{1:T})$,
\[
I_{\mathrm{op}}(X;O):=H(O)-H(O\mid X).
\]
\end{definition}

\begin{theorem}[Viotropic Wall]
There exists $\kappa>0$ such that
\[
H_{\mathrm{extractable}}>\kappa\,TC(T)\;\Rightarrow\; I_{\mathrm{op}}(X;O)=0 .
\]
Thus beyond capacity, internal structure becomes observationally null.
\end{theorem}

\section{Examples}

\paragraph{Finite Model Theory.}
For bounded-degree cycle-expanders $G_n$, the number of FO$^k_r$ local types is finite.
Recovering a global parity requires $\Omega(n)$ bits, while any FO$^k$ interface has
$TC(T)=O(1)$, forcing regime exit.

\paragraph{Physical Measurement.}
A finite-temperature sensor erasing one bit per timestep dissipates $k_B\Theta\ln 2$.
Thus $TC(T)=T$. Reconstructing an $N$-bit state with $T<N$ violates capacity.

\section{Scope}

All results are regime statements: they apply under declared locality, energy, and
bandwidth constraints. Violations manifest as forbidden residuals.

\begin{thebibliography}{9}
\bibitem{Libkin} L.~Libkin, \emph{Elements of Finite Model Theory}, Springer, 2004.
\bibitem{Immerman} N.~Immerman, \emph{Descriptive Complexity}, Springer, 1999.
\bibitem{CoverThomas} T.~Cover and J.~Thomas, \emph{Elements of Information Theory}, Wiley, 2006.
\bibitem{Landauer} R.~Landauer, \emph{IBM J. Res. Dev.} 5 (1961), 183--191.
\bibitem{Bekenstein} J.~D.~Bekenstein, \emph{Phys. Rev. D} 23 (1981), 287--298.
\end{thebibliography}

\end{document}

