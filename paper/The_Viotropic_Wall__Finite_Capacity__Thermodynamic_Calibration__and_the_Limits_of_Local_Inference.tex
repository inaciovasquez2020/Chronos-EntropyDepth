\documentclass[11pt]{article}

\usepackage[T1]{fontenc}
\usepackage{lmodern}
\usepackage{amsmath,amssymb,amsthm}
\usepackage{geometry}
\usepackage{hyperref}

\geometry{margin=1in}

\title{The Viotropic Wall: Finite Capacity, Thermodynamic Calibration,\\
and the Limits of Local Inference}
\author{Inacio F. Vasquez}
\date{January 2026}

\newtheorem{definition}{Definition}
\newtheorem{theorem}{Theorem}
\newtheorem{assumption}{Assumption}

\begin{document}
\maketitle

\begin{abstract}
Recent critiques of machine reasoning systems emphasize a recurring phenomenon:
increasing internal complexity does not translate into improved global understanding.
We formalize this observation as a universal rigidity principle.
Under three axioms—finite transcript capacity, thermodynamic calibration,
and informational monotonicity—we prove the existence of a sharp boundary,
the \emph{Viotropic Wall}, beyond which additional internal structure becomes
operationally invisible to any local observer.
The result applies uniformly across computational, physical, and biological
inference systems and yields a diagnostic protocol (IC3) for detecting regime
violations.
\end{abstract}

\section{Introduction and Motivation}

Local inference dominates modern science: algorithms refine hypotheses through
partial observations, physical instruments sample systems through bounded
interfaces, and biological agents adapt via limited sensory channels.
Yet in each domain the same obstruction appears: certain global structures
cannot be reliably inferred regardless of internal sophistication.

In finite model theory this manifests as locality theorems and bounded-variable
logics. In computation it appears as transcript and memory bounds.
In physics it is enforced by thermodynamic limits on information flow.
This paper isolates the common invariant underlying these phenomena.

\section{Interfaces and Transcripts}

\begin{definition}[Interface]
An interface mediates interaction between an unknown system state $X$
and an observer, producing a transcript $Y_{1:T}$ via admissible
observation rules.
\end{definition}

\begin{definition}[Transcript Capacity]
The transcript capacity of an interface over horizon $T$ is
\[
TC(T) := \sup I(X; Y_{1:T}).
\]
\end{definition}

\section{Axioms}

\begin{assumption}[Finite Capacity (FC)]
For any declared admissible regime, $TC(T) < \infty$ and
$TC(T) = O(T)$.
\end{assumption}

\begin{assumption}[Landauer Calibration]
If the interface dissipates heat at rate $\dot Q(t)$ at temperature $\Theta$, then
\[
TC(T) \le \frac{1}{k_B \Theta \ln 2}
\int_0^T \dot Q(t)\,dt .
\]
\end{assumption}

\begin{assumption}[Informational Monotonicity (Mon)]
For any admissible observer,
\[
H_{\mathrm{extractable}} \le TC(T).
\]
\end{assumption}

\section{The Viotropic Wall}

\begin{definition}[Operational Mutual Information]
For any observable $O$ derived from an admissible transcript, define
\[
I_{\mathrm{op}}(X;O) := H(O) - H(O \mid X).
\]
\end{definition}

\begin{theorem}[Viotropic Wall]
There exists $\kappa > 0$ such that for any admissible regime,
\[
\mathrm{Dem}_\eta(T) > \kappa \cdot TC(T)
\quad \Longrightarrow \quad
I_{\mathrm{op}}(X;O) = 0 .
\]
\end{theorem}

\begin{proof}
By Informational Monotonicity,
$H_{\mathrm{extractable}} \le TC(T)$.
If $\mathrm{Dem}_\eta(T) > \kappa TC(T)$, then no admissible transcript
can encode sufficient information to resolve $X$ to accuracy $\eta$.
Any observable $O$ derived from the transcript is therefore statistically
independent of $X$, yielding $I_{\mathrm{op}}(X;O)=0$.
\end{proof}

\section{IC3 Diagnostic Protocol}

We define the \emph{Information--Capacity Calorimeter} (IC3)
as a forensic test of admissibility:

\begin{enumerate}
\item Measure empirical $TC(\tilde T)$ from transcript statistics.
\item Measure thermodynamic capacity $TC_{\mathrm{therm}}(T)$ from heat dissipation.
\item Enforce monotonicity: $H_{\mathrm{extractable}} \le TC(\tilde T)$.
\item Test demand: $\mathrm{Dem}_\eta(T) > \kappa TC(\tilde T)$.
\item Probe outcome: estimate $I_{\mathrm{op}}(X;O)$.
\end{enumerate}

If $I_{\mathrm{op}} > 0$ under steps (1)--(4), a regime exit is certified.

\section{Discussion}

The Viotropic Wall is an informational event horizon.
Beyond it, internal structure may exist but cannot influence any admissible
observation.
The principle unifies locality theorems in logic, transcript bounds in
computation, and entropy limits in physics.

\section{Conclusion}

Under minimal and physically interpretable axioms, global inference is sharply
bounded.
Over-demanded information is not merely noisy; it is operationally extinct.

\begin{thebibliography}{9}

\bibitem{Libkin}
L.~Libkin,
\emph{Elements of Finite Model Theory},
Springer, 2004.

\bibitem{Immerman}
N.~Immerman,
\emph{Descriptive Complexity},
Springer, 1999.

\bibitem{CoverThomas}
T.~Cover and J.~Thomas,
\emph{Elements of Information Theory},
Wiley, 2006.

\bibitem{Landauer}
R.~Landauer,
Irreversibility and heat generation in the computing process,
\emph{IBM Journal of Research and Development} 5 (1961).

\end{thebibliography}

\end{document}

