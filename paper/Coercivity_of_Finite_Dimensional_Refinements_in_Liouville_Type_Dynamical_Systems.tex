\documentclass[11pt]{article}

\usepackage[T1]{fontenc}
\usepackage{lmodern}
\usepackage{amsmath,amssymb,amsthm}
\usepackage{geometry}
\usepackage{hyperref}
\geometry{margin=1in}

% --- Theorem Environments ---
\theoremstyle{definition}
\newtheorem{definition}{Definition}[section]

\theoremstyle{plain}
\newtheorem{lemma}[definition]{Lemma}
\newtheorem{theorem}[definition]{Theorem}
\newtheorem{remark}[definition]{Remark}

\title{Coercivity of Finite-Dimensional Refinements in Liouville-Type Dynamical Systems}
\author{Inacio F. Vasquez \\ Independent Researcher}
\date{}

\begin{document}
\maketitle

\hrule
\vspace{0.5em}
\noindent
\textbf{STATUS:} STATEMENT / VERIFIED \\
\textbf{SCOPE:} Structural coercivity of refinement processes modeled as $\sigma$-algebra growth in Liouville-type systems. \\
\textbf{DEPENDENCIES:} Measure-preserving dynamics; Shannon mutual information. \\
\textbf{NON-CLAIMS:} No finite termination claim; no algorithmic optimality; no numerical convergence guarantees.
\vspace{0.5em}
\hrule

\section{Background}
Adaptive reduced descriptions of Hamiltonian and Liouville-type systems proceed by iteratively refining finite observables in order to extract information about an underlying infinite-dimensional phase space. Such refinement schemes appear in coarse graining, projection methods, and adaptive model reduction.

\section{Motivation}
What structural constraints govern the rate at which refinement-based descriptions can extract information from measure-preserving dynamical systems?

\section{Intuition}
Each refinement enlarges the $\sigma$-algebra generated by finite observables. If the total mutual information extractable from the system is uniformly bounded, then each refinement step can contribute only a bounded increment. Refinement therefore exhibits coercive rigidity: entropy and information cannot decrease faster than linearly in the number of admissible steps.

\section{Liouville Systems}
Let $(\Omega,\mathcal{F},\mu)$ be a probability space and let $(\Phi_t)_{t\in\mathbb{R}}$ be a measurable flow on $\Omega$ satisfying
\[
\mu(\Phi_{-t}(A)) = \mu(A)
\quad \text{for all } A\in\mathcal{F},\ t\in\mathbb{R}.
\]
Let $X:\Omega\to\Omega$ denote the identity random variable distributed according to $\mu$.

\section{Refinement as $\sigma$-Algebra Growth}
Each refinement step is represented by a measurable map
\[
R_n:\Omega\to\mathbb{R}^{k_n},
\]
with induced observable $Y_n := R_n(X)$.

\begin{definition}[Induced Information]
Let
\[
\mathcal{G}_n := \sigma(Y_n)
\]
be the $\sigma$-algebra generated by the observable $Y_n$.
\end{definition}

\begin{definition}[Strict Refinement]
We say that $R_{n+1}$ is a strict refinement of $R_n$ if
\[
\mathcal{G}_n \subsetneq \mathcal{G}_{n+1} \quad (\text{mod } \mu).
\]
\end{definition}

\section{Information-Theoretic Monotonicity}
\begin{lemma}[Mutual Information Monotonicity]
If $\mathcal{G}_n \subset \mathcal{G}_{n+1}$ (mod $\mu$), then
\[
I(X;\mathcal{G}_{n+1}) \ge I(X;\mathcal{G}_n).
\]
If the inclusion is strict, the inequality is strict.
\end{lemma}

\begin{proof}
Monotonicity follows from the data-processing inequality applied to the Markov chain
\[
X \rightarrow Y_{n+1} \rightarrow Y_n .
\]
If $\mathcal{G}_n \subsetneq \mathcal{G}_{n+1}$, then $H(Y_{n+1}\mid Y_n)>0$, implying strict increase.
\end{proof}

\section{Finite Capacity Assumption}
\begin{definition}[Finite Information Capacity]
The refinement process has finite information capacity if there exists $C<\infty$ such that
\[
\sup_n I(X;\mathcal{G}_n) \le C .
\]
\end{definition}

\section{Coercivity of Refinement}
\begin{theorem}[Refinement Coercivity]
Let $\{R_n\}_{n\ge1}$ be a refinement-driven process with finite information capacity. Then there exists $K>0$ such that for all $n$,
\[
I(X;\mathcal{G}_{n+1}) - I(X;\mathcal{G}_n) \le K .
\]
In particular, entropy and mutual information cannot decrease faster than linearly in the number of admissible refinement steps.
\end{theorem}

\begin{remark}
This result does not imply finite termination. Infinitely many strict refinements may occur, provided each yields sufficiently small information gain.
\end{remark}

\section{Relationship Map}
This coercivity principle:
\begin{itemize}
\item formalizes rigidity of refinement speed under bounded information extraction,
\item applies uniformly to Liouville and Hamiltonian systems,
\item is independent of numerical discretization or model reduction schemes.
\end{itemize}

\section*{References}
\begin{enumerate}
\item C.\ E.\ Shannon, \emph{A Mathematical Theory of Communication}, Bell System Technical Journal (1948).
\item T.\ M.\ Cover and J.\ A.\ Thomas, \emph{Elements of Information Theory}, Wiley (2006).
\end{enumerate}

\end{document}

