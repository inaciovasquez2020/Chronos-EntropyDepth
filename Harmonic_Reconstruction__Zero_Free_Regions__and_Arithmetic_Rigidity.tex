\documentclass[11pt]{article}

\usepackage[T1]{fontenc}
\usepackage{lmodern}
\usepackage{amsmath,amssymb,amsthm}
\usepackage{geometry}
\usepackage{hyperref}

\geometry{margin=1in}

% --- Theorem Environments ---
\theoremstyle{definition}
\newtheorem{definition}{Definition}[section]

\theoremstyle{plain}
\newtheorem{theorem}[definition]{Theorem}
\newtheorem{lemma}[definition]{Lemma}

\theoremstyle{remark}
\newtheorem{remark}[definition]{Remark}

\title{Harmonic Reconstruction, Zero--Free Regions, and Arithmetic Rigidity}
\author{Inacio F. Vasquez\\Independent Researcher}
\date{February 2026}

\begin{document}
\maketitle

\begin{abstract}
We construct a Harmonic Reconstruction operator associated with the von Mangoldt
explicit formula and show that a classical zero--free region for the Riemann zeta function
implies a strict contraction bound on its prime--side component when restricted to
Paley--Wiener test spaces of sufficiently small bandwidth. This yields uniform coercivity,
stability under truncation, uniqueness of reconstruction, and an arithmetic rigidity
principle.
\end{abstract}

\section{Preliminaries}

Let $\zeta(s)$ denote the Riemann zeta function and let $\rho=\beta+i\gamma$ range over
its nontrivial zeros. We assume the classical zero--free region
\begin{equation}
\Re(s)\ge 1-\frac{A}{\log(|t|+2)} ,
\end{equation}
which is unconditional.

Let
\[
\psi(x)=\sum_{n\le x}\Lambda(n)
\]
denote the Chebyshev function.

\begin{theorem}[Explicit Prime Number Theorem under ZFR]
Assuming the zero--free region above, there exist explicit constants $B(A),c(A)>0$ such that
\[
|\psi(x)-x|\le B(A)\,x\,e^{-c(A)\sqrt{\log x}},\qquad x\ge2 .
\]
\end{theorem}

\section{Prime Error Kernel}

Define
\[
E(y)=\psi(e^{y})-e^{y}.
\]

\begin{lemma}[Weighted Integrability]
\[
\int_0^{\infty}|E(y)|e^{-y}\,dy<\infty .
\]
\end{lemma}

\begin{definition}[Arithmetic Constant]
Define
\[
C_{\mathrm{arith}}(A)=2\int_0^{\infty}|E(y)|e^{-y}\,dy .
\]
\end{definition}

\section{Test Space}

Fix $\Omega>0$.

\begin{definition}[Paley--Wiener Test Space]
Let
\[
\mathcal U=PW_\Omega\cap H^1(\mathbb R),\qquad u(0)=0 .
\]
\end{definition}

\begin{lemma}[Bernstein Inequality]
For all $u\in PW_\Omega$,
\[
\|u'\|_{L^2(\mathbb R)}\le \Omega\|u\|_{L^2(\mathbb R)} .
\]
\end{lemma}

Let $D\subset\mathcal U$ be any finite--dimensional subspace. All operators act on the
quotient Hilbert space $\mathcal U/D$.

\section{Harmonic Reconstruction Operator}

\begin{definition}[Prime Quadratic Form]
Define
\[
\langle u,Ku\rangle=\int_0^{\infty} E(y)\,d(|u(y)|^2),
\]
interpreted as a Stieltjes integral.
\end{definition}

\begin{lemma}[Boundedness and Symmetry]
The operator $K$ is bounded and self--adjoint on $\mathcal U/D$.
\end{lemma}

Define the Harmonic Reconstruction operator
\[
\mathcal O_{\mathrm{HR}}=I-K .
\]

\section{Contraction Bound}

\begin{lemma}[Contraction]
For all $u\in\mathcal U/D$,
\[
|\langle u,Ku\rangle|\le \Omega\,C_{\mathrm{arith}}(A)\,\|u\|^2 .
\]
\end{lemma}

\section{Critical Bandwidth and Coercivity}

\begin{definition}[Critical Bandwidth]
\[
\Omega^\ast(A)=\frac{1}{C_{\mathrm{arith}}(A)} .
\]
\end{definition}

\begin{theorem}[Uniform Coercivity]
If $\Omega<\Omega^\ast(A)$, then
\[
\langle u,\mathcal O_{\mathrm{HR}}u\rangle
\ge (1-\Omega C_{\mathrm{arith}}(A))\|u\|^2 .
\]
\end{theorem}

\section{Stability Under Truncation}

Let $\mathcal O_T$ denote the operator obtained by truncating the explicit formula to
zeros with $|\gamma|\le T$.

\begin{theorem}[Truncation Stability]
There exists $C=C(A)>0$ such that
\[
\|\mathcal O_{\mathrm{HR}}-\mathcal O_T\|\le C\frac{\log T}{T}.
\]
\end{theorem}

\section{Uniqueness and Arithmetic Rigidity}

\begin{theorem}[Uniqueness]
If $f_1,f_2\in\mathcal U/D$ produce identical spectral data, then $f_1=f_2$.
\end{theorem}

\begin{remark}[Arithmetic Rigidity]
The zero--free region enforces a finite rigidity window $\Omega<\Omega^\ast(A)$ below
which prime--side perturbations are uniquely detectable.
\end{remark}

\section*{References}

\begin{enumerate}
\item J.~B.~Rosser and L.~Schoenfeld, \emph{Illinois J. Math.} 6 (1962).
\item H.~Kadiri, arXiv:math/0510570.
\item S.~Faber and M.~Mossinghoff, \emph{Math. Comp.} 91 (2022).
\item E.~C.~Titchmarsh, \emph{The Theory of the Riemann Zeta--Function}, Oxford (1986).
\item T.~Kato, \emph{Perturbation Theory for Linear Operators}, Springer (1966).
\end{enumerate}

\end{document}

