\documentclass[11pt]{article}

\usepackage[T1]{fontenc}
\usepackage{lmodern}
\usepackage{amsmath,amssymb,amsthm}
\usepackage{geometry}
\usepackage{hyperref}

\geometry{margin=1in}

\theoremstyle{definition}
\newtheorem{definition}{Definition}[section]

\theoremstyle{plain}
\newtheorem{theorem}{Theorem}[section]

\theoremstyle{remark}
\newtheorem{remark}{Remark}[section]

\title{Thermodynamic Limits on Empirical Discovery in Finite Information Systems}
\author{Inacio F. Vasquez\\Independent Researcher}
\date{February 2026}

\begin{document}
\maketitle

\begin{abstract}
We propose a thermodynamic limit to catalog-based scientific discovery. In any finite
physical system subject to irreversible information processing, the marginal rate of
empirical discovery must decay to zero. The result follows from the finiteness of the
configuration space of discoverable objects together with a strictly positive physical
cost of information processing. We validate the prediction empirically using publicly
available infrared survey data from the NEOWISE mission.
\end{abstract}

\section{Physical Motivation}

Scientific discovery is a physical process. Observations require energy, data storage,
computation, and irreversible information processing. By Landauer’s principle, any
logically irreversible manipulation of information incurs a minimum thermodynamic
cost. Moreover, any finite physical system admits a bounded information capacity in
the sense of Bekenstein.

These facts suggest that empirical discovery itself may be subject to a fundamental
thermodynamic limit.

\section{Background and Intuition}

The mechanism is a compactness effect. In any finite configuration space, repeated
sampling necessarily exhausts distinguishable states. Early observations recover coarse
objects; later observations must search increasingly rare configurations.

Thermodynamics prevents arbitrarily cheap exploration: each refinement step requires
irreversible information processing with nonvanishing cost.

\section{Formal Setting}

Let $C$ be a finite configuration space of physically distinguishable discoverable
objects within a fixed observational regime.

Let $\Delta O_k$ denote the number of genuinely new objects discovered at step $k$,
and let $\Delta I_k$ denote the physical information processed at that step. Define the
discovery efficiency
\[
\Gamma_k = \frac{\Delta O_k}{\Delta I_k}.
\]

\subsection{Operational Admissibility}

A discovery is operationally admissible only if it results in a physically recorded
distinction produced by a finite sequence of localized interactions and irreversible
information-processing steps.

\section{Main Theorem}

\begin{theorem}[Discovery Saturation]
Let $C$ be finite and suppose each discovery step incurs a strictly positive physical
information cost $\Delta I_k \ge I_0 > 0$. Then
\[
\lim_{k\to\infty} \Gamma_k = 0.
\]
\end{theorem}

\begin{proof}
Since $C$ is finite,
\[
\sum_{k=1}^{\infty} \Delta O_k \le |C|,
\]
hence $\Delta O_k \to 0$ as $k\to\infty$.

By irreversible information processing, $\Delta I_k \ge I_0 > 0$ for all $k$. Therefore
\[
\Gamma_k = \frac{\Delta O_k}{\Delta I_k} \le \frac{\Delta O_k}{I_0} \longrightarrow 0.
\]
\end{proof}

\section{Interpretation}

Diminishing empirical returns are not sociological. They are a thermodynamic
consequence of finite distinguishability and irreducible information-processing cost.

\section{Empirical Validation: NEOWISE}

We analyze publicly available infrared survey data from NEOWISE and AllWISE.
Using a signal-to-noise threshold $\tau=7$, define
\[
N_k = \{d \in D_k : \max(\mathrm{SNR}_{W1},\mathrm{SNR}_{W2}) \ge \tau\},
\qquad
I_k = \log_2 |N_k|.
\]
Let $\Delta O_k$ denote the number of genuinely new moving objects identified at step
$k$. The empirical proxy
\[
\Gamma^{\mathrm{emp}}_k = \frac{\Delta O_k}{I_k}
\]
is observed to decrease over time, consistent with the theorem.

\section{Conclusion}

Empirical discovery obeys a universal thermodynamic saturation law. In any finite
domain, marginal discovery efficiency must vanish asymptotically.

\begin{thebibliography}{9}

\bibitem{Landauer}
R.~Landauer, Irreversibility and heat generation in the computing process,
\emph{IBM Journal of Research and Development} 5 (1961), 183--191.

\bibitem{Bekenstein}
J.~D.~Bekenstein, Black holes and entropy,
\emph{Physical Review D} 7 (1973), 2333--2346.

\bibitem{Wright}
E.~L.~Wright et al., The Wide-field Infrared Survey Explorer (WISE),
\emph{Astronomical Journal} 140 (2010), 1868--1881.

\bibitem{Mainzer}
A.~Mainzer et al., Initial performance of the NEOWISE reactivation mission,
\emph{Astrophysical Journal} 792 (2014), 30.

\bibitem{Shannon}
C.~E.~Shannon, A mathematical theory of communication,
\emph{Bell System Technical Journal} 27 (1948), 379--423.

\bibitem{Lloyd}
S.~Lloyd, Ultimate physical limits to computation,
\emph{Nature} 406 (2000), 1047--1054.

\end{thebibliography}

\end{document}

